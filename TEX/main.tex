% --------------------------------------------------------------
%                         Template HW
% --------------------------------------------------------------
\documentclass[a4paper, 12pt]{article} %draft = show box warnings
\usepackage[utf8]{inputenc} % Accept different input encodings [utf8]
%\usepackage[T1]{fontenc}    % Standard package for selecting font encodings
%\usepackage[a4paper, total={6.5in,10.2in}]{geometry} % Flexible and complete interface to document dimensions
\renewcommand{\baselinestretch}{1.3} 

% --------------------------------------------------------------
%                       Packages
% --------------------------------------------------------------
\usepackage{graphicx} % Import images

\usepackage{subcaption}
\usepackage{enumerate}

\usepackage[linktoc=all]{hyperref} % Create hyperlinks
\usepackage{float} % Good placement for float objects
\usepackage[export]{adjustbox} % Positioning figures
%\usepackage{amsmath,amsthm,amssymb} % American Mathematics Society facilities
%\usepackage[fancysections,titlepage,sectionmark]{polytechnique/polytechnique}

% --------------------------------------------------------------
%                         Languages
%			Change also the exercise environment
% --------------------------------------------------------------
%\usepackage[english]{babel} % Multilingual support [english]
\usepackage[french]{babel} % Multilingual support [french]
%\usepackage[brazilian]{babel} % Multilingual support [pt-BR]

% To use french quotes:
\usepackage[german=guillemets]{csquotes}

% More sizes for fonts:
\usepackage[]{moresize}

% --------------------------------------------------------------
%                         Fonts
% --------------------------------------------------------------
%\usepackage{lmodern} % Good looking T1 font
%\usepackage{mathpazo} % Hermann Zapf's Palatino font
%\usepackage{kpfonts} % Kepler font
%\usepackage{mathptmx} % Times New Roman Like Font
%\usepackage{eulervm} %  AMS Euler (eulervm) math font.

% --------------------------------------------------------------
%                       Begin Document
% --------------------------------------------------------------

\begin{document}

Cours de Physique de l'école Polytechnique

Par M. J. Jamin

Extrait - Pages 25 à 31 Volume 1

\textbf{MACHINE A DIVISER.} -- Cet instrument a pour organe essentiel une pièce de précision que l'on nomme \textit{vis micrométrique}. On la taille sur le contour d'un cylindre bien homogène de bronze ou d'acier fondu, long de 50 à 80 centimètres, et le constructeur, par des procédés mécaniques que nous n'avons pas à décrire, s'attache à obtenir, dans toute l'étendue du cylindre, un pas de vis constant et égal à 1 millimètre, c'est-à-dire, premièrement, que la distance entre deux filets consécutifs de la vis doit toujours être la même, secondement, toujours égale à 1 millimètre ; d'où il résulte que le nombre des filets compris le long d'une génératrice doit être égal au nombre de millimètres qui mesure sa longueur. Demander qu'une machine réalise absolument ces deux conditions, serait exiger une impossibilité ; mais il faut qu'elle y satisfasse très-sensiblement et qu'on puisse mesurer les inexactitudes qu'elle comporte : en attendant que nous disions comment on peut vérifier l'instrument, nous admettrons que la vis est parfaite.

A ses deux extrémités, le cylindre de la vis est saisi entre deux colliers \textbf{P} et \textbf{B} (\textit{fig.} \ref{fig1}), dans lesquels il peut tourner à frottement doux, sans avancer ni reculer, et une manivelle \textbf{A}

\begin{figure}[H]
	\begin{center}
		%\includegraphics[scale=0.6]{Images/topcompetiorsEua.png}
	\end{center}
	\caption{legenda}
	\label{fig1}
\end{figure}

que l'on tient à la main produit ce mouvement. La vis passe dans un écrou \textbf{Q} qui l'embrasse et qui ne peut tourner avec elle ; dès lors il avance ou recule quand elle tourne dans un sens ou dans le sens opposé ; en même temps il communique son mouvement à une règle d'acier \textbf{F} qui est fixée sur lui, et aussi à un burin \textbf{H} qui est attaché à cette règle ; le burin prend donc exactement le mouvement de l'écrou.

Il est clair que si la manivelle fait un tour entier, le burin avancera d'un pas de vis, c'est-'a-dire de 1 millimètre ; si elle décrit un dixième, un centième, un millième de tour, il marchera d'un dixième, d'un centième ou d'un millième de millimètre : il suffit donc de mesurer la fraction de tour parcourue par la manivelle pour avoir la fraction de millimètre franchie longitudinalement par le burin. A cet effet, la partie antérieure de la vis est munir d'un cercle \textbf{D} qui tourne avec elle et qui est divisé en 100 parties égales ; puis un index immobile \textbf{C}, fixé à la base de l'appareil, indique le déplacement du cercle.

Veut-on maintenant tracer des divisions équidistantes sur un tube de verre par exemple, on le dispose, comme la figure le montre, dans des coussinets où il est appuyé par deux cordes à boyau \textbf{L}, \textbf{K}, et dans lesquels il peut peut tourner sans avancer ou reculer ; on prend un burin de diamant, et on l'amène à l'un des bouts du tube où l'on trace la division initiale en tournant le tube d'une main et appuyant sur le burin de l'autre. Ensuite on décrit avec la manivelle un arc de \textit{n} divisons, ce qui fait marcher le burin de $\frac{n}{100}$ de millimètre, et l'on trace un second trait ; on répète ensuite la même opération jusqu'à la fin du tube.

Telle était la machine à diviser dans sa simplicité primitive, complète théoriquement, mais laissant beaucoup à désirer sous le rapport de la commodité dans l'emploi qu'on en faisait. On verra dans la \textit{fig.} \ref{fig2} un appareil plus perfectionné. La base \textbf{M} est en fonte et constitue un chemin de fer dont les rails supérieurs sont bien rabotés ; la vis se voit en \textbf{F}, et l'écrou qu'elle mer en mouvement est lié à la plaque \textbf{G}, qui avance avec lui en glissant sur les rails. Sur cette plaque on fixe l'objet \textbf{LL'} que l'on veut diviser. Quant au burin, il est placé en \textbf{H} et demeure fixe ; c'est l'objet qui se déplace et présente successivement ses divers points à l'action du traçoir. Pour plus de commodité, la manivelle placée en \textbf{A} imprime le mouvement à la vis par

\begin{figure}[H]
	\begin{center}
		%\includegraphics[scale=0.6]{Images/topcompetiorsEua.png}
	\end{center}
	\caption{legenda}
	\label{fig2}
\end{figure}

l'intermédiaire de deux roues dentées qui se rencontrent à angle droit. Telle que nous la décrivons, la nouvelle machine se comprend aisément ; mais il y a deux points particuliers sur lesquels nous allons insister : c'est la disposition du burin d'abord, et la mesure de la rotation de la vis ensuite.

On ne trace pas sur une règle que l'on divise des trits également allongés : le premier est long, les quatre suivants sont courts ; puis viennent quatre divisions courtes et une dixième qui est égale à la première. Or, dans l'ancienne machine, c'était à la main de l'opérateur qu'on laissait le soin d'aligner convenablement les traits, cela exigeait de l'habileté et une attention continuelle, sans produire toute la régularité possible ; la nouvelle disposition du burin, représentée en perspective dans la \textit{fig.} \ref{fig2} et en profil (\textit{fig.} \ref{fig3}), charge un mécanisme spécial de ce soin. On tient à la main le petit crochet \textbf{U}, on le tire d'abord vers soi en le soulevant ; ensuite on le pousse en appuyant légèrement, et le tracelet \textbf{H} pénètre dans la plaque à diviser où il marque le trait : pour donner à ce trait la longueur convenable, il suffit donc de limiter par des butoirs la course des pièces qui portent le burin.

Il y a au-dessus du burin une roue \textbf{IVX} qui peut tourner autour d'un axe fixe ; cette roue est composée de deux plaques circulaires, l'une \textbf{I} qui est dentée sur son contour, l'autre \textbf{VX}

\begin{figure}[H]
	\begin{center}
		%\includegraphics[scale=0.6]{Images/topcompetiorsEua.png}
	\end{center}
	\caption{legenda}
	\label{fig3}
\end{figure}

est entaillée d'échancrures alternativement profondes et peu creuses, séparées par des espaces qui forment e contour circulaire de la roue. Au moment où l'on tire le crochet \textbf{U}, une pièce saillante \textbf{X} avance vers la roue, pénètre dans une échancrure et, arrivée au fond, termine le mouvement du burin. Quand après cela on pousse le tracelet, il marche jusqu'à ce que l'on rencontre un butoir \textbf{T} qui ne permet pas d'aller plus loin.

Mais pendant que ce mouvement se fait, un crochet \textbf{R}, qui engrène dans la roue extérieure, la fait tourner d'une dent, déplace les échancrures de la roue voisine \textbf{V}, et quand ensuite on tire de nouveau le burin, la pièce \textbf{X} rencontre, non l'échancrure qui s'est abaissée, mais le contour extérieur qui a pris sa place ; la course est donc moins longue et le trait plus court que précédemment. A chaque mouvement, la même rotation des roues directrices se produit, et quand arrive la cinquième division, la saillie \textbf{X} pénètre dans une seconde échancrure qui allonge le trait, mais qui, étant moins profonde que la première, le fait moins long. En résumé, l'opérateur n'aura pas à s'occuper de la longueur de ses traits : ils seront bien alignés, et toutes les divisions multiples de 5 ou de 10 se reconnaîtront par des longueurs spéciales.

Le deuxième mécanisme que nous allons étudier est légitimé par le même besoin de simplifier la manœuvre en la confiant tout entière à l'appareil. Quand on opérait avec l'ancienne machine, il fallait entre le tracé de chaque division faire tourner la vis du même angle ; cela exigeait à chaque fois une petite opération d'arithmétique. On tournait, par exemple, de la division 0 à la division 12, ensuite de 12 à 24, puis de 24 à 36 : tout cela exigeait du du travail d'esprit, et il fallait à chaque fois se préoccuper d'arrêter la roue divisée à la division voulue, sans la dépasser tout en l'atteignant. C'est toute cette peine et toutes ces causes d'inexactitude que nous allons éviter.

La vis micrométrique \textbf{M} (\textit{fig.} \ref{fig4}) s'appuie sur le collet \textbf{NN}, que l'on a figuré démonté, c'est là qu'elle tourne sans déplacement; 

\begin{figure}[H]
	\begin{center}
		%\includegraphics[scale=0.6]{Images/topcompetiorsEua.png}
	\end{center}
	\caption{legenda}
	\label{fig4}
\end{figure}

elle se prolonge ensuite par une roue à rochet \textbf{R} et se termine par un axe rodé \textbf{A}.

Autour de cet axe \textbf{A} est une pièce \textbf{CCDE} que la figure montre en coupe, elle reçoit le mouvement de la manivelle par les roues d'angle \textbf{B}, \textbf{B'}, elle porte un cercle \textbf{E} large et épais, sur le contour extérieur duquel est teillée une vis dont nous verrons bientôt lusage : il est évident que cette pièce, étant folle sur l'axe \textbf{A}, peut tourner autour de lui sans faire mouvoir la vis.

Mais il y a un ressort \textbf{UF}, fixé d'une part sur le contour du cercle \textbf{E} et venant appuyer de l'autre sur les dents de la roue à rochet : si par la manivelle on fait tourner le cercle de \textbf{F} vers \textbf{U}, le ressort glisse sur les dents sans entraîner la roue, et la vis demeure immobile ; mais quand le mouvement se fait de \textbf{U} vers \textbf{F}, le ressort s'engage dans les dents, les chasse devant lui et imprime à la vis une rotation égale à celle qu'il a reçue. Par ce mécanisme, la vis micrométrique cesse de pouvoir marcher dans les deux sens ; tout mouvement qui va de \textbf{U} vers \textbf{F} lui est transmis ; tout mouvement qui est dirigé de \textbf{F} vers \textbf{U} la laisse en repos : on peut tourner la maivelle dans un sens sans produire aucune action ; mais dans le sens inverse, elle communique sa rotation à la vis \textbf{M}.

Cela posé, nous ferons remarquer un pignon denté \textbf{K} placé près du cercle \textbf{E}, et qui s'engrène avec la vis tracée sur ce cercle ; quand celui-ci fait un tour, le pignon marche d'une dent, dans un sens ou dans l'autre. Or il y a sur le cercle un butoir \textbf{I} et sur le pignon un arrêt \textbf{Z}, et le mouvement de deux pièces que nous examinons s'arrête quand ce butoir \textbf{I} et cet arrêt \textbf{Z} se rencontrent : c'est là un point de départ fixe, et arrivés à ce moment, nous traçons une division. Nous faisons alors mouvoir le cercle \textbf{E} de \textbf{U} vers \textbf{F} ; la vis marche, le pignon denté se déplace angulairement de \textbf{X} vers \textbf{Z}, et il arrive bientôt qu'un second arrêt \textbf{X} du pignon rencontre un deuxième butoir \textbf{D} du cercle ; alors le mouvement s'arrête, et la vis a marché d'une quantité déterminée par les positions respectives des deux systèmes d'arrêts : on marque un deuxième trait. Ensuite on tourne la manivelle en sens contraire, ce qui ne déplace pas la vis, mais ramène en contact les deux butoirs \textbf{I} et \textbf{Z}, comme ils y étaient au point de départ, et l'on peut recommencer indéfiniment sans avoir à se préoccuper de mesurer la rotation. Disons pour terminer que les arrêts \textbf{I} et \textbf{Z} sont fixes et les deux autres \textbf{X} et \textbf{D} mobiles ; c'est en plaçant convenablement ceux-ci, au moyen d'une graduation faite sur le contour \textbf{CC} du cercle, qu'on règle la fraction de tour que l'on fait d'un arrêt à l'autre.

Sans insister aujourd'hui sur les nombreux usages de la machine à diviser, revenons à la questions qui nous a amené à la décrire. Nous avons une règle en métal qui a la longueur du mètre, et nous voulons la diviser en millimètres. Si la machine à diviser que nous possédons était parfaite, cette opération serait bien simple : on réglerait la course des butoirs de manière à faire tourner la vis d'une circonférence complète à chaque fois, les divisions tracées seraient égales à 1 millimètre, et la première étant à l'un des bouts du mètre, la millième arriverait nécessairement à l'autre extrémité ; mais dans la pratique cela n'arrive pas, et il faut commencer par étudier la machine avant de l'employer. On fixe le mètre sur la plaque mobile parallèlement à la vis, on place au-dessus un microscope portant des fils croisés, et l'on vise l'extrémité, puis on fait marcher la machine ; le mètre se déplace, et l'on compte le nombre de tours qu'il faut faire pour amener l'autre extrémité sous la croisée des fils du microscope. Cette opération ne peut se faire en une seule fois, parce que la vis micrométrique n'a pas un mètre de longueur ; mais on fractionne le mètre en parties successives sur lesquelles on fait séparément cette opération. On trouve généralement que le nombre total des pas de vis contenus dans la longueur du mètre n'est pas égal à 1000 ;par example, il sera égal à 998 ; cela voudra dire qu'un pas de vis est égal à $\frac{1000}{998}$ millimètres ou $1^{mm}$, 002, et par suite si l'on veut tracer 1000 division sur la règle, il faudra les espacer d'une fraction de pas de vis représentée par 0,998 et faire tourner la vis de cette fraction de tour entre deux divisions consécutives ; dans ce cas, on est sûr que la millième division tombe exactement au second bout du mètre, si la division initiale coïncide avec le premier.

Grâce à cette machine, nous pouvons, comme on le voit, diviser des règles en millimètres et adapter ces règles à tous les appareils qui devront mesurer les longueurs ; mais cela ne suffit pas encore : il faut pouvoir pousser plus loin la division du mètre et apprécier des fractions de millimètre. Il y a pour remplir ce but un appareil fort simple, c'est le vernier.

\end{document}
